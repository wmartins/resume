% resume.tex
%
% (c) 2002 Matthew Boedicker <mboedick@mboedick.org> (original author) http://mboedick.org
% (c) 2003-2007 David J. Grant <davidgrant-at-gmail.com> http://www.davidgrant.ca
% (c) 2007-2014 Todd C. Miller <Todd.Miller@courtesan.com> http://www.courtesan.com/todd
%
% This work is licensed under the Creative Commons Attribution-ShareAlike 3.0 Unported License. To view a copy of this license, visit http://creativecommons.org/licenses/by-sa/3.0/ or send a letter to Creative Commons, 171 Second Street, Suite 300, San Francisco, California, 94105, USA.

\documentclass[letterpaper,11pt]{article}

%-----------------------------------------------------------
\usepackage[utf8]{inputenc}
\usepackage[empty]{fullpage}
\usepackage{color}
\usepackage{hyperref}
\definecolor{mygrey}{gray}{0.85}
\raggedbottom
\raggedright
\setlength{\tabcolsep}{0in}

\hypersetup{
    colorlinks=true,
    linkcolor=blue,
    filecolor=blue,      
    urlcolor=blue
}

% Adjust margins to 0.5in on all sides
\addtolength{\oddsidemargin}{-0.5in}
\addtolength{\evensidemargin}{-0.5in}
\addtolength{\textwidth}{1.0in}
\addtolength{\topmargin}{-0.5in}
\addtolength{\textheight}{1.0in}

%-----------------------------------------------------------
%Custom commands
\newcommand{\resitem}[1]{\item #1 \vspace{-2pt}}
\newcommand{\resheading}[1]{{\large \colorbox{mygrey}{\begin{minipage}{\textwidth}{\textbf{#1 \vphantom{p\^{E}}}}\end{minipage}}}}
\newcommand{\ressubheading}[4]{
\begin{tabular*}{7.0in}{l@{\extracolsep{\fill}}r}
    \textbf{#1} & #2 \\
    \textit{#3} & \textit{#4} \\
\end{tabular*}\vspace{-6pt}}
%-----------------------------------------------------------

\begin{document}

    \begin{tabular*}{7.5in}{l@{\extracolsep{\fill}}r}
    \textbf{\large William H. Martins} & +55 51 9626 1149 \\
    Avenida Bento Gonçalves, 1515      & \href{mailto:wwwhmartins@gmail.com}{wwwhmartins@gmail.com} \\
    Porto Alegre - Brasil              & \href{https://github.com/wmartins}{github.com/wmartins} \\
                                       & \href{https://linkedin.com/in/wwwhmartins}{linkedin.com/in/wwwhmartins}
    \end{tabular*}
    \\
    \vspace{0.1in}

    \resheading{Education}
    \begin{itemize}
        \item
            \ressubheading
                {Pontifícia Universidade Católica do Rio Grande do Sul}
                {Porto Alegre - Rio Grande do Sul - Brasil}
                {Bacharelado em Ciência de Computação}
                {Fev. 2012 - Dez. 2016}

        \item
            \ressubheading
                {Instituto Federal Sul-rio-grandense}
                {Charqueadas - Rio Grande do Sul - Brasil}
                {Curso Técnico Integrado em Informática}
                {Fev. 2008 - Dez. 2011}
    \end{itemize}

    \resheading{Experiência}
    \begin{itemize}
        \item
            \ressubheading
                {ADP - Automatic Data Processing}
                {Porto Alegre - Rio Grande do Sul - Brasil}
                {Software Developer}
                {Março 2015 - Atualmente}
            \begin{itemize}
                \resitem{Parte do time de desenvolvimento, sendo responsavel tanto por código \textit{front-end} quando \textit{back-end}.}
                \resitem{Ajudou a estabelcer um modelo de \textit{git flow} efetivo que permitiu que o time desenvolvesse novas funcionalidades enquanto entregava código em produção com confiança.}
                \resitem{Responsável por refatorar testes de uma aplicação \textit{back-end}, que ajudou o time a desenvolver nossas funcionalidades com confiança.}
                \resitem{Critou um pequeno e efetivo \textit{framework} de interfaces, usado para prototipar e desenvolver novas telas.}
            \end{itemize}

        \item
            \ressubheading
                {dEx digital}
                {Porto Alegre - Rio Grande do Sul - Brasil}
                {Front-End Developer}
                {October 2013 - February 2015}
            \begin{itemize}
                \resitem{Responsável pelo desenvolvimento de sites responsivos para os clientes da empresa.}
                \resitem{Criou um novo \textit{workflow}, utilizando ferramentas e tecnologias novas.}
                \resitem{Ajudou o time de \textit{back-end} no desenvolvimento de algumas funcionalidades.}
            \end{itemize}

        \item
            \ressubheading
                {Ockan Networks}
                {Porto Alegre - Rio Grande do Sul - Brasil}
                {Front-End Developer}
                {January 2012 - October 2013}
            \begin{itemize}
                \resitem{Parte do time de desenvolvimento/análise, sendo responsável por obter requisitos dos sistemas e construír soluções \textit{web} para eles.}
                \resitem{Introduziu um novo \textit{framework} que fez o time ser muito mais produtivo.}
            \end{itemize}

        \item
            \ressubheading
                {TMW Telecom}
                {Arroio dos Ratos - Rio Grande do Sul - Brasil}
                {Web Developer}
                {January 2011 - December 2011}
            \begin{itemize}
                \resitem{Parte do time de desenvolvimento/análise, sendo responsável por obter requisitos dos sistemas e construír soluções \textit{web} para eles.}
                \resitem{Criou uma nova aplicação para gerenciar o \textit{workflow} da empresa.}
            \end{itemize}
    \end{itemize}

    \resheading{Habilidades}
    \begin{description}
        \item[Linguagens]
            HTML, CSS, JavaScript, NodeJS, C/C++, Python, Java, PHP, \LaTeX.
        \item[Bancos de Dados]
            PostgreSQL, MySQL, Oracle.
        \item[Tecnologias para \textit{Web}]
            PostCSS, webpack, SASS, LESS, ReactJS.
        \item[Outros]
            Linux, Git, Bash, Docker, GitHub.
    \end{description}

    \resheading{Projetos de Código Aberto}
    \begin{itemize}
        \item
            \ressubheading
                {Mongration}
                {\href{https://github.com/awapps/mongration}{github.com/awapps/mongration}}
                {MongoDB migration framework}
                {}
    \end{itemize}

\end{document}
